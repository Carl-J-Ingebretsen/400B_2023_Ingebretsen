\documentclass[
  journal=Notes,
  manuscript=Homework,
  year=2022,
  volume=1,
]{dgl-journal}

\usepackage[nopatch]{microtype}
\usepackage{booktabs}


\title{ASTR400B Homework 3}

\author{Carl Ingebretsen}
\email[Author 1]{carljingebretsen@arizona.edu}

%\keywords{keyword1, keyword2, keyword3} 

\begin{document}

%\begin{abstract}
%Lorem ipsum dolor sit amet, consectetur adipiscing elit, sed do eiusmod tempor incididunt ut labore et dolore magna aliqua. Ante in nibh mauris cursus mattis molestie a iaculis at. Varius vel pharetra vel turpis nunc eget lorem. Gravida rutrum quisque non tellus orci ac. Augue interdum velit euismod in pellentesque massa placerat duis. Ac tortor dignissim convallis aenean et. Tincidunt lobortis feugiat vivamus at.
%\end{abstract}

\section{Question 1:}

\begin{table}[]
    \centering
    \begin{tabular}{||c|c|c|c|c|c||}
        Galaxy name & Halo Mass $10^{12} M_\odot$ & Disk Mass $10^{12} M_\odot$ & Bulge Mass $10^{12} M_\odot$ & Total Mass $10^{12} M_\odot$ & $f_{baryon}$\\
        \hline
        $M_{31}$ & $1.921$ & $0.120$ & $0.01905$ & $2.060$ & $0.0675$ \\
        \hline
        $M_{33}$ & $0.187$ & $0.00930$ & $N\backslash A$ & $0.196$ & $0.0475$ \\
        \hline
        Milky Way & $1.975$ & $0.0750$ & $0.0100$ & $2.060$ & $0.0413$ \\
        \hline
        Local Group & & & & $4.316$ & $0.0541$
      \end{tabular}
    \caption{This table summarizes the masses and baryon fractions of each of the main components of the local group and their major parts. These parts of the galaxies are the halo mass, which is made of dark matter, the disk mass, and the bulge mass. }
    \label{mass_table}
\end{table}

In this simulation, the Milky Way and the Andromeda galaxy have the same total mass. The dark matter halos of each galaxy dominate their total masses.

\section{Question 2:}
The Andromeda Galaxy has greater stellar masses in its disk and bulge than the Milky Way. This would lead me to conclude that the Andromeda Galaxy would be the more luminous of the two.

\section{Question 3:}
According to the parameters of this simulation, the Milky Way has a greater dark matter mass than the Andromeda Galaxy. This is somewhat surprising since the Andromeda Galaxy has a greater stellar mass. The ratio of the Milky Way's dark matter to Andromeda's dark matter mass is $1.026$.

\section{Question 4:}
The Milky Way has a baryon fraction of $4.13\%$, Andromeda has a baryon fraction of $6.75\%$ and the Triangulum Galaxy has a baryon fraction of $4.75\%$. These values are about a quarter of the universal baryon fraction of $16\%$. The universe has large amounts of gas between galaxies and bound to galaxy clusters. This gas can be observed in the X-rays. The galaxies lack this extra gas that contributes to the baryon fraction of the universe. This leads to the lower values observed in the galaxies.

\end{document}